\documentclass[12pt,a4paper]{report}

% Packages
\usepackage[utf8]{inputenc}
\usepackage[english]{babel}
\usepackage{geometry}
\usepackage{graphicx}
\usepackage{xcolor}
\usepackage{listings}
\usepackage{tikz}
\usepackage{tcolorbox}
\usepackage{amsmath}
\usepackage{amssymb}
\usepackage{hyperref}
\usepackage{fancyhdr}
\usepackage{enumitem}
\usepackage{tabularx}
\usepackage{booktabs}
\usepackage{float}

% Page setup
\geometry{margin=1in}
\pagestyle{fancy}
\fancyhf{}
\fancyhead[L]{\leftmark}
\fancyhead[R]{\thepage}

% Colors
\definecolor{examdef}{RGB}{0, 102, 204}
\definecolor{concept}{RGB}{0, 153, 76}
\definecolor{example}{RGB}{204, 102, 0}
\definecolor{warning}{RGB}{204, 0, 0}

% Custom boxes
\newtcolorbox{exambox}[1]{
  colback=examdef!5!white,
  colframe=examdef!75!black,
  title=#1,
  fonttitle=\bfseries
}

\newtcolorbox{conceptbox}[1]{
  colback=concept!5!white,
  colframe=concept!75!black,
  title=#1,
  fonttitle=\bfseries
}

\newtcolorbox{examplebox}[1]{
  colback=example!5!white,
  colframe=example!75!black,
  title=#1,
  fonttitle=\bfseries
}

% Hyperref setup
\hypersetup{
    colorlinks=true,
    linkcolor=blue,
    filecolor=magenta,      
    urlcolor=cyan,
    pdftitle={Computer Networks - Complete Study Guide},
    pdfpagemode=FullScreen,
}

% Title page
\title{
    \huge\textbf{Computer Networks} \\
    \Large Complete Study Guide \\
    \vspace{1cm}
    \large Chapters 1-6: Conceptual Understanding \\
    \vspace{0.5cm}
    \normalsize Final Exam Preparation
}
\author{Comprehensive Notes with Diagrams and Examples}
\date{\today}

\begin{document}

\maketitle

\tableofcontents
\newpage

% ============================================
% CHAPTER 1: Computer Networks and Internet
% ============================================

\chapter{Computer Networks and the Internet}

\section{Nuts-and-Bolts Description}

\begin{exambox}{Definition: The Internet}
The Internet is a worldwide network infrastructure connecting billions of computing devices (hosts/end systems) through communication links, packet switches, and standardized protocols.
\end{exambox}

\subsection{Key Components}

\subsubsection{Hosts / End Systems}
\begin{itemize}
    \item Computing devices at network edge
    \item Examples: PCs, smartphones, servers, IoT devices
    \item Run network applications
    \item Connected via communication links
\end{itemize}

\begin{conceptbox}{Conceptual Understanding}
\textbf{Why "End Systems"?} \\
Devices sit at the EDGE of network. Applications run here, not in network core. Think of them as the starting and ending points of all communication.
\end{conceptbox}

\subsubsection{Communication Links}
\begin{itemize}
    \item Physical media connecting devices
    \item Types: Fiber optic, coaxial cable, copper wire, wireless
    \item Transmission rate measured in bps (bits per second)
    \item Examples: 100 Mbps Ethernet, 1 Gbps Fiber
\end{itemize}

\subsubsection{Packet Switches}
\begin{itemize}
    \item \textbf{Routers}: Network layer, forward packets based on IP
    \item \textbf{Link-layer switches}: Access networks, forward based on MAC
    \item Intermediate devices that forward packets toward destination
\end{itemize}

\subsection{Internet Service Providers (ISPs)}

\begin{table}[H]
\centering
\begin{tabular}{|l|p{10cm}|}
\hline
\textbf{ISP Type} & \textbf{Description} \\
\hline
Tier 1 & Global backbone providers (AT\&T, Sprint) \\
Tier 2 & Regional ISPs connecting to Tier 1 \\
Tier 3 & Local access ISPs (last mile to homes) \\
Mobile & Cellular networks (4G, 5G) \\
\hline
\end{tabular}
\caption{ISP Hierarchy}
\end{table}

\subsection{Protocols and Standards}

\begin{itemize}
    \item \textbf{TCP/IP}: Core Internet protocols
    \item \textbf{IETF}: Internet Engineering Task Force
    \item \textbf{RFCs}: Request for Comments ($\sim$9000 documents)
    \item \textbf{IEEE}: Standards for Ethernet (802.3), WiFi (802.11)
\end{itemize}

% ============================================
\section{Access Networks}

\subsection{DSL (Digital Subscriber Line)}

\begin{exambox}{Definition: DSL}
DSL is broadband access using existing telephone lines with frequency division multiplexing to provide simultaneous data and voice services.
\end{exambox}

\textbf{How DSL Works:}
\begin{enumerate}
    \item Home DSL modem connects to phone line
    \item Splitter separates voice (0-4 kHz) and data (4 kHz-1 MHz)
    \item DSLAM at central office separates signals
    \item Data routed to Internet, voice to telephone network
\end{enumerate}

\textbf{Characteristics:}
\begin{itemize}
    \item Asymmetric: Download $>$ Upload
    \item Dedicated line (not shared)
    \item Distance limited: 5-10 miles from CO
    \item Typical speeds: 24-52 Mbps down, 3.5-16 Mbps up
\end{itemize}

\subsection{Cable Internet}

\begin{exambox}{Definition: Cable Internet}
Cable Internet uses existing cable TV infrastructure with hybrid fiber-coax (HFC) to provide high-speed Internet access.
\end{exambox}

\textbf{Architecture:}
\begin{itemize}
    \item Fiber optic from head end to neighborhood junction
    \item Coaxial cable to individual homes
    \item Cable modem at home, CMTS at head end
\end{itemize}

\textbf{Key Feature: Shared Broadcast Medium}
\begin{examplebox}{Bandwidth Sharing Example}
10 homes share 100 Mbps coaxial cable:
\begin{itemize}
    \item All 10 active: $\sim$10 Mbps each
    \item 1 active: Full 100 Mbps
    \item Bandwidth varies with neighborhood usage
\end{itemize}
\end{examplebox}

\subsection{FTTH (Fiber To The Home)}

\begin{itemize}
    \item Optical fiber directly to homes
    \item Extremely high speeds: 100 Mbps - 10 Gbps
    \item PON architecture: OLT → Splitter → ONTs
    \item Low signal loss, immune to interference
\end{itemize}

\subsection{Comparison of Access Technologies}

\begin{table}[H]
\centering
\small
\begin{tabular}{|l|c|c|c|c|}
\hline
\textbf{Technology} & \textbf{Speed} & \textbf{Shared?} & \textbf{Infrastructure} & \textbf{Cost} \\
\hline
DSL & 24-52 Mbps & No & Phone lines & Low \\
Cable & 40-1000 Mbps & Yes & Coax + Fiber & Medium \\
FTTH & 100 Mbps-10 Gbps & Somewhat & Fiber & High \\
5G Fixed & 100+ Mbps & Wireless & None & Medium \\
\hline
\end{tabular}
\caption{Access Network Technologies}
\end{table}

% ============================================
\section{Packet Switching}

\subsection{Store-and-Forward Transmission}

\begin{exambox}{Definition: Store-and-Forward}
A packet switch must receive the \textbf{entire} packet before it can begin transmitting the packet on the outbound link.
\end{exambox}

\textbf{End-to-End Delay Formula:}
\begin{equation}
d_{end-to-end} = N \times \frac{L}{R}
\end{equation}

Where:
\begin{itemize}
    \item $N$ = number of links
    \item $L$ = packet length (bits)
    \item $R$ = transmission rate (bps)
\end{itemize}

\begin{examplebox}{Delay Calculation Example}
Packet: $L = 1500$ bytes $= 12,000$ bits \\
Link speed: $R = 1$ Mbps $= 1,000,000$ bps \\
Number of links: $N = 3$

\textbf{Solution:}
\begin{align*}
\text{Delay per link} &= \frac{L}{R} = \frac{12,000}{1,000,000} = 0.012 \text{ sec} = 12 \text{ ms} \\
\text{Total delay} &= 3 \times 12 = 36 \text{ ms}
\end{align*}
\end{examplebox}

\subsection{Queuing and Packet Loss}

\textbf{Output Queue:}
\begin{itemize}
    \item Packets wait in queue before transmission
    \item Queue length depends on arrival rate vs. transmission rate
    \item Finite buffer size
\end{itemize}

\textbf{Packet Loss:}
\begin{itemize}
    \item Occurs when buffer is full
    \item Arriving packet must be dropped
    \item Causes: Arrival rate $>$ Transmission rate
\end{itemize}

% ============================================
\section{Circuit Switching}

\begin{exambox}{Definition: Circuit Switching}
A dedicated communication path (circuit) is established for the entire duration of a communication session with reserved resources.
\end{exambox}

\subsection{Multiplexing Methods}

\subsubsection{FDM (Frequency Division Multiplexing)}
\begin{itemize}
    \item Divides frequency bandwidth into bands
    \item Each circuit gets dedicated frequency band
    \item All transmit simultaneously
    \item Example: Radio stations on different frequencies
\end{itemize}

\subsubsection{TDM (Time Division Multiplexing)}
\begin{itemize}
    \item Divides time into frames and slots
    \item Each circuit gets specific time slots
    \item Circuits take turns transmitting
    \item Example: Each connection gets 1/4 of time in 4-way TDM
\end{itemize}

\subsection{Packet vs Circuit Switching}

\begin{table}[H]
\centering
\begin{tabular}{|l|p{5cm}|p{5cm}|}
\hline
\textbf{Feature} & \textbf{Packet Switching} & \textbf{Circuit Switching} \\
\hline
Resources & On-demand & Reserved (dedicated) \\
Setup & No setup & Connection setup required \\
Sharing & Statistical multiplexing & TDM or FDM \\
Efficiency & High & Low (idle waste) \\
Delay & Variable (queuing) & Fixed (no queuing) \\
Suited For & Bursty data & Voice, video \\
\hline
\end{tabular}
\caption{Packet vs Circuit Switching Comparison}
\end{table}

% ============================================
\chapter{Application Layer}

\section{Principles of Network Applications}

\subsection{Client-Server Architecture}

\begin{exambox}{Definition: Client-Server}
Client-server architecture features a dedicated always-on server that services requests from multiple client hosts, where clients don't directly communicate with each other.
\end{exambox}

\textbf{Server Characteristics:}
\begin{itemize}
    \item Always-on (24/7 availability)
    \item Fixed, well-known IP address
    \item Powerful hardware
    \item Located in data centers
\end{itemize}

\textbf{Client Characteristics:}
\begin{itemize}
    \item Intermittently connected
    \item Dynamic IP addresses
    \item Initiate communication
    \item Don't communicate directly with other clients
\end{itemize}

\subsection{P2P Architecture}

\begin{exambox}{Definition: Peer-to-Peer}
P2P architecture features minimal reliance on dedicated servers where intermittently connected peers communicate directly, with each peer acting as both client and server.
\end{exambox}

\textbf{Self-Scalability:}
\begin{itemize}
    \item More peers = More resources
    \item Each peer provides and consumes resources
    \item Cost-effective (no server infrastructure)
    \item Examples: BitTorrent, Bitcoin
\end{itemize}

\subsection{Sockets and Port Numbers}

\begin{conceptbox}{Socket Analogy}
\textbf{Socket} = Door of your house \\
\textbf{Process} = Person inside \\
\textbf{Network} = Street outside \\
\\
Messages sent through door (socket) to postal service (transport layer).
\end{conceptbox}

\textbf{Port Number Ranges:}
\begin{itemize}
    \item \textbf{Well-Known (0-1023):} HTTP (80), HTTPS (443), SMTP (25), DNS (53)
    \item \textbf{Registered (1024-49151):} MySQL (3306), PostgreSQL (5432)
    \item \textbf{Dynamic (49152-65535):} Temporary client ports
\end{itemize}

% ============================================
\section{HTTP (HyperText Transfer Protocol)}

\begin{exambox}{Definition: HTTP}
HTTP is the application layer protocol of the World Wide Web that defines how clients request web pages and servers transfer them, using TCP as transport.
\end{exambox}

\subsection{HTTP Characteristics}

\textbf{Stateless Protocol:}
\begin{itemize}
    \item Server maintains NO information about past requests
    \item Each request is independent
    \item Simplifies server design, improves scalability
    \item State maintained using cookies
\end{itemize}

\subsection{Persistent vs Non-Persistent Connections}

\textbf{Non-Persistent HTTP (HTTP/1.0):}
\begin{itemize}
    \item New TCP connection for each object
    \item High overhead (multiple handshakes)
    \item Response time: $2 \times RTT + $ transmission time per object
\end{itemize}

\textbf{Persistent HTTP (HTTP/1.1):}
\begin{itemize}
    \item Multiple objects over same connection
    \item Reduced overhead
    \item Faster page loading
    \item Default mode
\end{itemize}

\begin{examplebox}{RTT Comparison}
Page with 1 base HTML + 10 images:

\textbf{Non-Persistent:}
\begin{itemize}
    \item 11 TCP connections
    \item Total: $22 \times RTT$ + transmission
\end{itemize}

\textbf{Persistent:}
\begin{itemize}
    \item 1 TCP connection
    \item Total: $12 \times RTT$ + transmission
    \item 45\% faster!
\end{itemize}
\end{examplebox}

\subsection{HTTP Message Format}

\textbf{Request Message:}
\begin{verbatim}
GET /index.html HTTP/1.1
Host: www.example.com
Connection: close
User-agent: Mozilla/5.0
Accept-language: en
\end{verbatim}

\textbf{Response Message:}
\begin{verbatim}
HTTP/1.1 200 OK
Date: Mon, 10 Dec 2025 12:00:00 GMT
Server: Apache/2.4
Content-Length: 1024
Content-Type: text/html

<html>...</html>
\end{verbatim}

\textbf{Common Status Codes:}
\begin{itemize}
    \item \textbf{200 OK:} Success
    \item \textbf{301 Moved Permanently:} Redirect
    \item \textbf{400 Bad Request:} Client error
    \item \textbf{404 Not Found:} Resource doesn't exist
    \item \textbf{500 Internal Server Error:} Server error
\end{itemize}

\subsection{Cookies}

\begin{exambox}{Definition: Cookies}
Cookies are small pieces of data stored on the client side that enable websites to track users and maintain session state despite HTTP being stateless.
\end{exambox}

\textbf{Four Components:}
\begin{enumerate}
    \item Cookie header in HTTP response (Set-Cookie)
    \item Cookie header in HTTP request (Cookie)
    \item Cookie file on client's browser
    \item Backend database on server
\end{enumerate}

\subsection{Web Caching}

\textbf{Benefits:}
\begin{itemize}
    \item Reduced response time
    \item Reduced traffic on access link
    \item Reduced load on origin server
    \item Better user experience
\end{itemize}

\textbf{Conditional GET:}
\begin{verbatim}
Request:
GET /page.html HTTP/1.1
If-Modified-Since: Mon, 10 Dec 2025 10:00:00 GMT

Response (Not Modified):
HTTP/1.1 304 Not Modified

Response (Modified):
HTTP/1.1 200 OK
Last-Modified: Mon, 10 Dec 2025 14:00:00 GMT
[new content]
\end{verbatim}

% ============================================
\section{Email and SMTP}

\subsection{Email Components}

\begin{itemize}
    \item \textbf{User Agents:} Outlook, Gmail, Thunderbird
    \item \textbf{Mail Servers:} Store mailboxes, implement SMTP
    \item \textbf{SMTP:} Simple Mail Transfer Protocol (Port 25)
\end{itemize}

\subsection{SMTP Basics}

\textbf{SMTP Dialog:}
\begin{verbatim}
S: 220 mail.example.com
C: HELO sender.com
S: 250 Hello sender.com
C: MAIL FROM: <alice@sender.com>
S: 250 OK
C: RCPT TO: <bob@example.com>
S: 250 OK
C: DATA
S: 354 Send message, end with .
C: Subject: Hello
C: 
C: This is the message body.
C: .
S: 250 Message accepted
C: QUIT
S: 221 Closing connection
\end{verbatim}

\subsection{Mail Access Protocols}

\begin{itemize}
    \item \textbf{IMAP:} Manage emails remotely on server
    \item \textbf{HTTP:} Web-based email (Gmail, Yahoo)
\end{itemize}

% ============================================
\chapter{Transport Layer}

\section{Transport Layer Services}

\begin{exambox}{Definition: Transport Layer}
The transport layer provides logical communication between application processes running on different hosts, extending network layer's host-to-host delivery to process-to-process delivery.
\end{exambox}

\subsection{Multiplexing and Demultiplexing}

\textbf{Multiplexing (Sender):}
\begin{itemize}
    \item Gather data from multiple sockets
    \item Add transport headers (create segments)
    \item Pass to network layer
\end{itemize}

\textbf{Demultiplexing (Receiver):}
\begin{itemize}
    \item Receive segments from network layer
    \item Examine port numbers
    \item Deliver to correct socket
\end{itemize}

\subsection{UDP Socket Identification}

\textbf{Two-tuple:} (Destination IP, Destination Port)

\begin{itemize}
    \item Same destination $\rightarrow$ Same socket
    \item Source port used as "return address"
    \item Connectionless
\end{itemize}

\subsection{TCP Socket Identification}

\textbf{Four-tuple:} (Source IP, Source Port, Dest IP, Dest Port)

\begin{itemize}
    \item Each connection gets unique socket
    \item Welcoming socket listens for new connections
    \item Connection sockets handle established connections
\end{itemize}

% ============================================
\section{UDP (User Datagram Protocol)}

\begin{exambox}{Definition: UDP}
UDP is a minimalist, connectionless transport protocol providing multiplexing/demultiplexing and basic error checking without reliability, flow control, or congestion control.
\end{exambox}

\subsection{UDP Header (8 bytes)}

\begin{table}[H]
\centering
\begin{tabular}{|c|c|}
\hline
Source Port (16 bits) & Destination Port (16 bits) \\
\hline
Length (16 bits) & Checksum (16 bits) \\
\hline
\multicolumn{2}{|c|}{Data / Payload} \\
\hline
\end{tabular}
\caption{UDP Header Format}
\end{table}

\subsection{UDP Checksum}

\begin{conceptbox}{Checksum Calculation}
\textbf{Sender:}
\begin{enumerate}
    \item Treat segment as 16-bit words
    \item Add all words (wrap overflow)
    \item Take 1's complement
    \item Result = Checksum
\end{enumerate}

\textbf{Receiver:}
\begin{enumerate}
    \item Add all words including checksum
    \item If result = all 1s $\rightarrow$ No error
    \item Else $\rightarrow$ Error detected
\end{enumerate}
\end{conceptbox}

% ============================================
\section{TCP (Transmission Control Protocol)}

\begin{exambox}{Definition: TCP}
TCP is a connection-oriented, reliable transport protocol providing full-duplex, point-to-point communication with flow control, congestion control, and in-order delivery.
\end{exambox}

\subsection{TCP Characteristics}

\begin{itemize}
    \item \textbf{Connection-oriented:} 3-way handshake
    \item \textbf{Reliable:} Retransmits lost segments
    \item \textbf{In-order delivery:} Sequence numbers
    \item \textbf{Flow control:} Prevent receiver overflow
    \item \textbf{Congestion control:} Prevent network congestion
\end{itemize}

\subsection{TCP Header (20-60 bytes)}

\textbf{Key Fields:}
\begin{itemize}
    \item Sequence Number (32 bits)
    \item Acknowledgment Number (32 bits)
    \item Receive Window (16 bits)
    \item Checksum (16 bits)
    \item Flags: SYN, ACK, FIN, RST, PSH, URG
\end{itemize}

\subsection{RTT Estimation}

\begin{align}
\text{EstimatedRTT} &= (1-\alpha) \times \text{EstimatedRTT} + \alpha \times \text{SampleRTT} \\
\alpha &= 0.125 \\
\\
\text{DevRTT} &= (1-\beta) \times \text{DevRTT} + \beta \times |\text{SampleRTT} - \text{EstimatedRTT}| \\
\beta &= 0.25 \\
\\
\text{TimeoutInterval} &= \text{EstimatedRTT} + 4 \times \text{DevRTT}
\end{align}

\subsection{Three-Way Handshake}

\begin{enumerate}
    \item \textbf{Client $\rightarrow$ Server:} SYN (Seq=x)
    \item \textbf{Server $\rightarrow$ Client:} SYN-ACK (Seq=y, ACK=x+1)
    \item \textbf{Client $\rightarrow$ Server:} ACK (Seq=x+1, ACK=y+1)
\end{enumerate}

\subsection{Flow Control}

\begin{equation}
\text{rwnd} = \text{RcvBuffer} - [\text{LastByteRcvd} - \text{LastByteRead}]
\end{equation}

Sender ensures:
\begin{equation}
\text{LastByteSent} - \text{LastByteAcked} \leq \text{rwnd}
\end{equation}

% ============================================
\chapter{Network Layer - Data Plane}

\section{Overview}

\subsection{Forwarding vs Routing}

\begin{table}[H]
\centering
\begin{tabular}{|l|p{5cm}|p{5cm}|}
\hline
\textbf{Feature} & \textbf{Forwarding} & \textbf{Routing} \\
\hline
Scope & Local (single router) & Global (entire network) \\
Function & Move packet to output & Determine end-to-end path \\
Speed & Nanoseconds & Milliseconds to seconds \\
Implementation & Hardware & Software \\
Plane & Data plane & Control plane \\
\hline
\end{tabular}
\caption{Forwarding vs Routing}
\end{table}

% ============================================
\section{Router Architecture}

\subsection{Components}

\begin{enumerate}
    \item \textbf{Input Ports:} Receive packets, lookup, forward
    \item \textbf{Switching Fabric:} Connect input to output ports
    \item \textbf{Output Ports:} Queue, schedule, transmit packets
    \item \textbf{Routing Processor:} Control plane operations
\end{enumerate}

\subsection{Switching Fabric Types}

\begin{table}[H]
\centering
\begin{tabular}{|l|c|c|c|}
\hline
\textbf{Type} & \textbf{Speed} & \textbf{Parallelism} & \textbf{Cost} \\
\hline
Memory & Slowest & None & Low \\
Bus & Medium & None & Medium \\
Crossbar & Fastest & Yes & High \\
\hline
\end{tabular}
\caption{Switching Fabric Comparison}
\end{table}

% ============================================
\section{IP Addressing}

\subsection{IPv4 Datagram Format}

\textbf{Key Fields:}
\begin{itemize}
    \item Version (4 bits): IPv4 = 4
    \item Header Length (4 bits): Typically 5 (20 bytes)
    \item Total Length (16 bits): Max 65,535 bytes
    \item TTL (8 bits): Hop limit
    \item Protocol (8 bits): TCP=6, UDP=17
    \item Source/Dest IP (32 bits each)
\end{itemize}

\subsection{CIDR (Classless Interdomain Routing)}

\begin{exambox}{CIDR Notation}
\textbf{Format:} a.b.c.d/x

\textbf{Example:} 192.168.1.0/24
\begin{itemize}
    \item First 24 bits: Network prefix
    \item Last 8 bits: Host part
    \item Total addresses: $2^8 = 256$
\end{itemize}
\end{exambox}

\subsection{DHCP (Dynamic Host Configuration Protocol)}

\textbf{4-Step Process:}
\begin{enumerate}
    \item \textbf{DHCP DISCOVER:} Client broadcasts
    \item \textbf{DHCP OFFER:} Server responds with IP
    \item \textbf{DHCP REQUEST:} Client accepts offer
    \item \textbf{DHCP ACK:} Server confirms assignment
\end{enumerate}

\subsection{NAT (Network Address Translation)}

\begin{conceptbox}{NAT Operation}
\textbf{Private network $\rightarrow$ Internet:}
\begin{itemize}
    \item Replace (source IP, source port)
    \item With (NAT IP, new port)
    \item Store in NAT translation table
\end{itemize}

\textbf{Internet $\rightarrow$ Private network:}
\begin{itemize}
    \item Look up (NAT IP, port) in table
    \item Replace with (private IP, original port)
\end{itemize}
\end{conceptbox}

% ============================================
\chapter{Network Layer - Control Plane}

\section{Routing Algorithms}

\subsection{Distance Vector vs Link State}

\begin{table}[H]
\centering
\small
\begin{tabular}{|l|p{5cm}|p{5cm}|}
\hline
\textbf{Feature} & \textbf{Distance Vector} & \textbf{Link State} \\
\hline
Algorithm & Bellman-Ford & Dijkstra \\
Knowledge & Neighbors only & Entire topology \\
Exchange & Routing tables & Link state packets \\
Updates & Periodic/Triggered & Event-driven \\
Convergence & Slow & Fast \\
Example & RIP, EIGRP & OSPF, IS-IS \\
\hline
\end{tabular}
\caption{Routing Algorithms Comparison}
\end{table}

% ============================================
\section{OSPF (Open Shortest Path First)}

\begin{exambox}{Definition: OSPF}
OSPF is a link-state intra-AS routing protocol using Dijkstra's algorithm to compute shortest paths within an autonomous system.
\end{exambox}

\subsection{OSPF Features}

\begin{itemize}
    \item \textbf{Security:} MD5 authentication
    \item \textbf{Multiple paths:} ECMP support
    \item \textbf{Hierarchical:} Area-based design
    \item \textbf{Multicast:} MOSPF extension
\end{itemize}

% ============================================
\section{BGP (Border Gateway Protocol)}

\begin{exambox}{Definition: BGP}
BGP is the inter-domain routing protocol enabling communication between autonomous systems through policy-based path selection.
\end{exambox}

\subsection{BGP Attributes}

\begin{itemize}
    \item \textbf{AS-PATH:} List of ASs traversed (loop prevention)
    \item \textbf{NEXT-HOP:} IP of next-hop router
    \item \textbf{LOCAL-PREF:} Local preference
    \item \textbf{MED:} Multi-Exit Discriminator
\end{itemize}

\subsection{Hot Potato Routing}

\begin{conceptbox}{Hot Potato Concept}
Minimize intra-AS cost to reach exit point, regardless of inter-AS path cost. "Get rid of packet ASAP!"
\end{conceptbox}

% ============================================
\chapter{Link Layer and LANs}

\section{Error Detection}

\subsection{Parity Checks}

\textbf{Even Parity:} Total 1s must be even
\textbf{Odd Parity:} Total 1s must be odd

\textbf{Limitation:} Cannot detect even-numbered bit errors

\subsection{Checksums}

\begin{enumerate}
    \item Sum all 16-bit words
    \item Wrap overflow
    \item Take 1's complement
    \item Receiver verifies by summing all (should be all 1s)
\end{enumerate}

\subsection{CRC (Cyclic Redundancy Check)}

\textbf{Process:}
\begin{enumerate}
    \item Treat data as polynomial
    \item Divide by generator polynomial (XOR)
    \item Remainder = CRC
    \item Append to data
    \item Receiver divides by same generator
    \item Zero remainder = No error
\end{enumerate}

% ============================================
\section{Multiple Access Protocols}

\subsection{Channel Partitioning}

\begin{itemize}
    \item \textbf{TDM:} Divide time into slots
    \item \textbf{FDM:} Divide frequency bands
    \item \textbf{CDMA:} Unique codes for each node
\end{itemize}

\subsection{Random Access}

\textbf{ALOHA Protocols:}
\begin{itemize}
    \item \textbf{Pure ALOHA:} Transmit immediately, 18\% efficiency
    \item \textbf{Slotted ALOHA:} Synchronized slots, 37\% efficiency
\end{itemize}

\textbf{CSMA Protocols:}
\begin{itemize}
    \item \textbf{CSMA/CD:} Collision detection (Ethernet)
    \item \textbf{CSMA/CA:} Collision avoidance (WiFi)
\end{itemize}

\subsection{Taking Turns}

\begin{itemize}
    \item \textbf{Polling:} Master node controls access
    \item \textbf{Token Passing:} Token circulates among nodes
    \item \textbf{Reservation:} Reserve time slots in advance
\end{itemize}

% ============================================
\section{MAC Addressing}

\begin{exambox}{Definition: MAC Address}
MAC address is a unique 48-bit physical address permanently assigned to each network adapter for identification in local networks.
\end{exambox}

\textbf{Format:} 6 bytes in hexadecimal (e.g., 1A:2B:3C:4D:5E:6F)

\subsection{ARP (Address Resolution Protocol)}

\textbf{Process:}
\begin{enumerate}
    \item Check ARP table for IP$\rightarrow$MAC mapping
    \item If not found, broadcast ARP query
    \item Destination responds with MAC address
    \item Update ARP table (with TTL)
    \item Send data using MAC address
\end{enumerate}

% ============================================
\section{VLANs}

\begin{exambox}{Definition: VLAN}
A VLAN is a logical grouping of devices into separate broadcast domains through switch configuration, providing traffic isolation.
\end{exambox}

\subsection{VLAN Benefits}

\begin{itemize}
    \item Traffic isolation
    \item Dynamic membership
    \item Better resource utilization
    \item Simplified management
\end{itemize}

\subsection{VLAN Trunking (802.1Q)}

\textbf{Trunk Port:} Carries traffic for multiple VLANs

\textbf{VLAN Tag (4 bytes):}
\begin{itemize}
    \item TPID (2 bytes): 0x8100
    \item TCI (2 bytes): PCP (3 bits) + DEI (1 bit) + VLAN ID (12 bits)
\end{itemize}

% ============================================
% Summary and Exam Tips
% ============================================

\chapter{Exam Preparation Summary}

\section{Key Formulas}

\begin{align}
\text{Transmission Delay} &= \frac{L}{R} \\
\text{Packet Switching Delay} &= N \times \frac{L}{R} \\
\text{EstimatedRTT} &= (1-\alpha) \times \text{EstimatedRTT} + \alpha \times \text{SampleRTT} \\
\text{TimeoutInterval} &= \text{EstimatedRTT} + 4 \times \text{DevRTT} \\
\text{rwnd} &= \text{RcvBuffer} - [\text{LastByteRcvd} - \text{LastByteRead}]
\end{align}

\section{Important Port Numbers}

\begin{table}[H]
\centering
\begin{tabular}{|l|c|l|}
\hline
\textbf{Protocol} & \textbf{Port} & \textbf{Description} \\
\hline
HTTP & 80 & Web traffic \\
HTTPS & 443 & Secure web \\
SMTP & 25 & Email sending \\
DNS & 53 & Name resolution \\
FTP & 20, 21 & File transfer \\
SSH & 22 & Secure shell \\
\hline
\end{tabular}
\caption{Common Port Numbers}
\end{table}

\section{Critical Comparisons}

\textbf{Must Know:}
\begin{itemize}
    \item Packet Switching vs Circuit Switching
    \item TCP vs UDP
    \item Distance Vector vs Link State
    \item OSPF vs BGP
    \item CSMA/CD vs CSMA/CA
    \item IPv4 vs IPv6
\end{itemize}

\section{Exam Tips}

\begin{enumerate}
    \item \textbf{Understand concepts}, not just memorize
    \item \textbf{Draw diagrams} for protocol operations
    \item \textbf{Practice calculations} (delay, throughput)
    \item \textbf{Know protocol headers} and fields
    \item \textbf{Explain "why"} mechanisms exist
    \item \textbf{Compare and contrast} similar protocols
\end{enumerate}

% ============================================
\end{document}
